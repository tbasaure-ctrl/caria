
\begin{table}[H]
\centering
\begin{threeparttable}
\caption{\textbf{Strategy Backtest: ASF Signals vs Benchmarks}}
\label{tab:strategy}
\begin{tabular}{lcccccc}
\toprule
\textbf{Strategy} & \textbf{CAGR} & \textbf{Volatility} & \textbf{Sharpe} & \textbf{Max DD} & \textbf{Calmar} \\
\midrule
Buy & Hold SPX & 33.5\% & 40.7\% & 0.82 & -56.3\% & 0.60 \\
60/40 Portfolio & 44.5\% & 23.9\% & 1.86 & -30.6\% & 1.46 \\
ASF Strategy & 32.9\% & 36.7\% & 0.90 & -54.9\% & 0.60 \\
VIX Strategy & 19.9\% & 24.0\% & 0.83 & -30.7\% & 0.65 \\

\bottomrule
\end{tabular}
\begin{tablenotes}[flushleft]
\footnotesize
\item \textit{Notes:} Backtest period 2000--2024. CAGR = Compound Annual Growth Rate. Max DD = Maximum Drawdown. Transaction costs of 10 bps per trade included. ASF Strategy adjusts equity exposure based on structural fragility signals; VIX Strategy uses volatility-based timing.
\end{tablenotes}
\end{threeparttable}
\end{table}
