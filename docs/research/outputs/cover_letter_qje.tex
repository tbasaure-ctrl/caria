% Cover Letter for Quarterly Journal of Economics
\documentclass[12pt]{letter}
\usepackage[margin=1in]{geometry}
\usepackage{setspace}
\usepackage{hyperref}

\onehalfspacing

\signature{Tom\'as Basaure Larra\'in\\Pontificia Universidad Cat\'olica de Chile\\Email: tbasaure@uc.cl}
\address{Tom\'as Basaure Larra\'in\\Department of Economics\\Pontificia Universidad Cat\'olica de Chile\\Santiago, Chile}

\begin{document}

\begin{letter}{Editorial Office\\The Quarterly Journal of Economics\\Harvard University}

\opening{Dear Editors,}

The enclosed manuscript, ``\textbf{When Stability Becomes Fragility: Phase Transitions in Financial Markets},'' is submitted for consideration for publication in the \textit{Quarterly Journal of Economics}.

\textbf{Summary.} This paper provides empirical evidence that the mapping from market structure to systemic tail risk is fundamentally nonlinear and regime-dependent, exhibiting characteristics of a phase transition. A critical connectivity threshold is identified at which the effect of structural fragility on future crash risk inverts sign---formalizing a transition from ``contagion'' dynamics to ``disintegration'' dynamics in financial risk propagation.

\textbf{Key Contributions.} The paper makes four main contributions to the literature:

\begin{enumerate}
    \item \textbf{Phase Transition Evidence:} First empirical documentation of a critical coupling threshold in financial markets below which structural fragility amplifies tail risk and above which the relationship inverts.
    
    \item \textbf{Novel State Variable:} Introduction of Accumulated Spectral Fragility (ASF), a state variable with memory that captures the persistence of low-dimensional market structure, addressing a key gap in existing systemic risk measures.
    
    \item \textbf{Rigorous Econometric Framework:} Application of Hansen (2000) threshold regression methodology with HAC-robust inference, bootstrap confidence intervals, Bayesian verification, surrogate data falsification, and extensive robustness analysis.
    
    \item \textbf{Policy Relevance:} A structural resolution to the volatility paradox with direct implications for macroprudential monitoring in an era of passive investing and tightly coupled markets.
\end{enumerate}

\textbf{Relevance to QJE.} The findings speak to fundamental questions in economics about endogenous risk, market stability, and the nonlinear nature of financial crises. The research bridges macroeconomic theory (Minsky's Financial Instability Hypothesis, the Brunnermeier-Sannikov volatility paradox) with empirical finance and complex systems, offering insights relevant to policymakers, financial regulators, and academic economists.

\textbf{Methodology.} The empirical analysis utilizes 35 years of global multi-asset data (1990--2024) and employs threshold regression estimation, HAC standard errors, bootstrap inference, Granger causality tests, and surrogate data falsification following Theiler et al. (1992). Results are robust across alternative specifications, connectivity metrics, and control variables.

\textbf{Declarations.} This manuscript has not been published elsewhere and is not under consideration by any other journal. A complete replication package including data and code will be made available upon acceptance. There are no conflicts of interest to declare.

The author is available to provide any additional information that may be required during the review process.

\closing{Respectfully submitted,}

\end{letter}
\end{document}



