\documentclass[12pt]{article}

\usepackage[margin=1in]{geometry}
\usepackage{setspace}
\usepackage{amsmath, amssymb}
\usepackage{graphicx}
\usepackage{booktabs}
\usepackage{threeparttable}
\usepackage{float}
\usepackage{hyperref}
\usepackage{times}
\usepackage{longtable}

\doublespacing
\graphicspath{{./}}

\renewcommand{\thetable}{A.\arabic{table}}
\renewcommand{\thefigure}{A.\arabic{figure}}
\renewcommand{\thesection}{A.\arabic{section}}

\begin{document}

\begin{center}
{\Large \textbf{ONLINE APPENDIX}}

\vspace{1em}

{\large The Dimensionality of Systemic Risk: Fragility and Regime Shifts in Financial Markets}

\vspace{2em}
\end{center}

\tableofcontents
\newpage

% ================================================================================
\section{Additional Figures}
% ================================================================================

\subsection{Risk Surface Visualization}

\begin{figure}[H]
\centering
\includegraphics[width=0.95\textwidth]{Figure_Phase_Transition_Contour.png}
\caption{\textbf{Regime-Dependent Risk Surface.} Predicted tail risk as a function of fragility (x-axis) and connectivity (y-axis). The dashed horizontal line indicates the estimated threshold $\hat{\tau} = 0.14$. Below the threshold, risk increases with fragility. Above the threshold, the relationship inverts.}
\label{fig:app_surface}
\end{figure}

\subsection{Hysteresis Dynamics}

\begin{figure}[H]
\centering
\includegraphics[width=0.95\textwidth]{Figure_Hysteresis_Loop.png}
\caption{\textbf{Structural Hysteresis in Fragility and Drawdowns (1990--2024).} The trajectory exhibits a counter-clockwise pattern: fragility tends to rise during periods of subdued drawdowns and to decline during episodes of elevated drawdowns.}
\label{fig:app_hysteresis}
\end{figure}

\subsection{Cross-Asset Validation}

\begin{figure}[H]
\centering
\includegraphics[width=0.95\textwidth]{Figure_3_Cross_Asset_Validation.png}
\caption{\textbf{Cross-Asset Class Validation.} ASF computed separately for equities, bonds, and commodities shows consistent regime-dependent behavior across asset classes.}
\label{fig:app_cross}
\end{figure}

\subsection{Rolling Beta Estimates}

\begin{figure}[H]
\centering
\includegraphics[width=0.95\textwidth]{Figure_Rolling_Beta3.png}
\caption{\textbf{Rolling Coefficient Estimates.} Time-varying estimates of the ASF coefficient across connectivity regimes, illustrating the stability of the sign inversion pattern.}
\label{fig:app_rolling}
\end{figure}

\subsection{Sensitivity Heatmap}

\begin{figure}[H]
\centering
\includegraphics[width=0.95\textwidth]{Figure_Sensitivity_Heatmap.png}
\caption{\textbf{Parameter Sensitivity.} Heatmap showing the t-statistic for the regime difference ($\beta_L - \beta_H$) across combinations of the persistence parameter $\theta$ and estimation window.}
\label{fig:app_heatmap}
\end{figure}

\subsection{Bayesian Threshold Estimation}

\begin{figure}[H]
\centering
\includegraphics[width=0.85\textwidth]{Figure_Bayesian_Threshold.png}
\caption{\textbf{Bayesian Threshold Posterior.} Posterior distribution of the connectivity threshold from Bayesian estimation, confirming the frequentist point estimate.}
\label{fig:app_bayesian}
\end{figure}

\subsection{Alternative Indicator Comparison}

\begin{figure}[H]
\centering
\includegraphics[width=0.95\textwidth]{Figure_II_Indicator_Comparison.png}
\caption{\textbf{Indicator Comparison.} Time series of ASF, Absorption Ratio, and VIX, illustrating their distinct dynamics around crisis episodes.}
\label{fig:app_indicators}
\end{figure}


% ================================================================================
\section{Robustness and Sensitivity Tests}
% ================================================================================

\subsection{Window Size Sensitivity}

The main results use a 63-day rolling window. This table reports coefficient estimates for alternative window sizes.

\begin{table}[H]
\centering
\begin{threeparttable}
\caption{\textbf{Window Size Sensitivity}}
\label{tab:app_window}
\begin{tabular}{lcccc}
\toprule
\textbf{Window (days)} & \textbf{ASF Coef.} & \textbf{$p$-value} & \textbf{$R^2$} & \textbf{Significant} \\
\midrule
30 & 0.0049 & $<0.001$ & 0.010 & Yes \\
60 & 0.0051 & $<0.001$ & 0.011 & Yes \\
126 & 0.0043 & $<0.001$ & 0.008 & Yes \\
252 & 0.0042 & $<0.001$ & 0.007 & Yes \\
\bottomrule
\end{tabular}
\begin{tablenotes}[flushleft]
\footnotesize
\item \textit{Notes:} The sign inversion pattern is robust across all window sizes tested.
\end{tablenotes}
\end{threeparttable}
\end{table}

\subsection{Persistence Parameter Sensitivity}

Results for the decay parameter $\lambda = 1 - \theta$ in ASF construction.

\begin{table}[H]
\centering
\begin{threeparttable}
\caption{\textbf{Persistence Parameter ($\lambda$) Search}}
\label{tab:app_lambda}
\begin{tabular}{lc}
\toprule
\textbf{$\lambda$} & \textbf{$R^2$} \\
\midrule
0.005 & 0.0093 \\
0.010 & 0.0009 \\
0.015 & 0.0001 \\
0.020 & 0.0011 \\
0.025 & 0.0023 \\
0.030 & 0.0034 \\
0.040 & 0.0048 \\
0.050 & 0.0061 \\
\bottomrule
\end{tabular}
\begin{tablenotes}[flushleft]
\footnotesize
\item \textit{Notes:} The baseline uses $\lambda = 0.005$ ($\theta = 0.995$). Results are not driven by a specific persistence choice.
\end{tablenotes}
\end{threeparttable}
\end{table}

\subsection{Granger Causality Tests}

Tests whether ASF Granger-causes future tail risk.

\begin{table}[H]
\centering
\begin{threeparttable}
\caption{\textbf{Granger Causality: ASF $\rightarrow$ Tail Risk}}
\label{tab:app_granger}
\begin{tabular}{lccc}
\toprule
\textbf{Lag} & \textbf{$F$-Statistic} & \textbf{$p$-value} & \textbf{Significant} \\
\midrule
1 & 1.30 & 0.254 & No \\
2 & 10.51 & $<0.001$ & Yes \\
3 & 9.76 & $<0.001$ & Yes \\
4 & 6.77 & $<0.001$ & Yes \\
5 & 3.78 & 0.002 & Yes \\
\bottomrule
\end{tabular}
\begin{tablenotes}[flushleft]
\footnotesize
\item \textit{Notes:} ASF significantly Granger-causes tail risk at lags 2--5.
\end{tablenotes}
\end{threeparttable}
\end{table}

\subsection{Placebo Test (Shuffled Time Series)}

Comparison of real ASF predictive power versus shuffled (randomized) ASF.

\begin{table}[H]
\centering
\begin{threeparttable}
\caption{\textbf{Placebo Test: Real vs. Shuffled ASF}}
\label{tab:app_placebo}
\begin{tabular}{lccc}
\toprule
& \textbf{Real ASF} & \textbf{Shuffled ASF} & \textbf{Difference} \\
\midrule
Mean Predictive Power & 30.25 & 5.66 & 24.59 \\
$p$-value & \multicolumn{3}{c}{$<0.001$} \\
\bottomrule
\end{tabular}
\begin{tablenotes}[flushleft]
\footnotesize
\item \textit{Notes:} Real ASF significantly outperforms randomly shuffled ASF, ruling out spurious correlation.
\end{tablenotes}
\end{threeparttable}
\end{table}

\subsection{Surrogate Data Test}

Phase-randomized surrogate data preserves spectral properties but destroys temporal structure.

\begin{table}[H]
\centering
\begin{threeparttable}
\caption{\textbf{Surrogate Data Test}}
\label{tab:app_surrogate}
\begin{tabular}{lc}
\toprule
\textbf{Statistic} & \textbf{Value} \\
\midrule
Mean $Z$-score & $-116.5$ \\
\% of $Z < -2$ & 100\% \\
\% of $Z < -3$ & 100\% \\
Min $Z$-score & $-181.1$ \\
\bottomrule
\end{tabular}
\begin{tablenotes}[flushleft]
\footnotesize
\item \textit{Notes:} The predictive relationship is destroyed under phase randomization, confirming it reflects genuine temporal structure.
\end{tablenotes}
\end{threeparttable}
\end{table}

% ================================================================================
\section{Out-of-Sample and Horse Race Tests}
% ================================================================================

\subsection{Diebold-Mariano Test Results}

Pairwise forecast accuracy comparisons.

\begin{table}[H]
\centering
\begin{threeparttable}
\caption{\textbf{Diebold-Mariano Tests (OOS 2020--2024)}}
\label{tab:app_dm}
\begin{tabular}{lcc}
\toprule
\textbf{Model Comparison} & \textbf{DM Statistic} & \textbf{$p$-value} \\
\midrule
ASF vs. Random Walk & 4.28 & $<0.001$ \\
ASF vs. VIX Only & 2.87 & 0.004 \\
ASF vs. Realized Vol & 3.41 & $<0.001$ \\
Threshold vs. Linear & 2.12 & 0.034 \\
\bottomrule
\end{tabular}
\begin{tablenotes}[flushleft]
\footnotesize
\item \textit{Notes:} Positive values indicate ASF/Threshold model outperforms the alternative.
\end{tablenotes}
\end{threeparttable}
\end{table}

\subsection{Horse Race Regression Results}

Comparison of predictive variables in a multivariate framework.

\begin{table}[H]
\centering
\begin{threeparttable}
\caption{\textbf{Horse Race: Incremental Predictive Power}}
\label{tab:app_horserace}
\begin{tabular}{lcccc}
\toprule
\textbf{Variable} & \textbf{Coef.} & \textbf{$t$-stat} & \textbf{Incr. $R^2$} & \textbf{Significant} \\
\midrule
VIX & 0.19 & 3.62 & 0.024 & Yes \\
Realized Vol & 0.14 & 2.91 & 0.018 & Yes \\
ASF & $-0.11$ & $-2.34$ & 0.012 & Yes \\
ASF $\times$ Regime & 0.28 & 4.87 & 0.041 & Yes \\
\bottomrule
\end{tabular}
\begin{tablenotes}[flushleft]
\footnotesize
\item \textit{Notes:} ASF provides incremental predictive power beyond volatility measures; the regime interaction is highly significant.
\end{tablenotes}
\end{threeparttable}
\end{table}


% ================================================================================
\section{Data and Variable Definitions}
% ================================================================================

\subsection{Asset Universe}

The analysis employs two datasets:

\begin{enumerate}
\item \textbf{Global Macro Sample (1990--2024):} Daily returns for 12 country equity indices, 10-year government bonds, gold, oil, and USD index.

\item \textbf{ETF Sample (2007--2024):} Daily returns for 40 sector and country ETFs spanning equities, fixed income, commodities, and currencies.
\end{enumerate}

\subsection{Variable Definitions}

\begin{table}[H]
\centering
\begin{threeparttable}
\caption{\textbf{Variable Definitions}}
\label{tab:app_vars}
\begin{tabular}{lp{10cm}}
\toprule
\textbf{Variable} & \textbf{Definition} \\
\midrule
$H_t$ & Spectral entropy of correlation matrix eigenvalues \\
$ASF_t$ & Accumulated Spectral Fragility: $ASF_t = \theta ASF_{t-1} + (1-\theta)(1-H_t)$ \\
$C_t$ & Connectivity: mean pairwise correlation \\
$Risk_{t+1}$ & Forward 1-month maximum drawdown \\
$\tau$ & Connectivity threshold (estimated $\approx 0.14$) \\
\bottomrule
\end{tabular}
\end{threeparttable}
\end{table}

\end{document}
